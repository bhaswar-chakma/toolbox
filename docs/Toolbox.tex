% Options for packages loaded elsewhere
\PassOptionsToPackage{unicode}{hyperref}
\PassOptionsToPackage{hyphens}{url}
%
\documentclass[
]{book}
\usepackage{amsmath,amssymb}
\usepackage{lmodern}
\usepackage{ifxetex,ifluatex}
\ifnum 0\ifxetex 1\fi\ifluatex 1\fi=0 % if pdftex
  \usepackage[T1]{fontenc}
  \usepackage[utf8]{inputenc}
  \usepackage{textcomp} % provide euro and other symbols
\else % if luatex or xetex
  \usepackage{unicode-math}
  \defaultfontfeatures{Scale=MatchLowercase}
  \defaultfontfeatures[\rmfamily]{Ligatures=TeX,Scale=1}
\fi
% Use upquote if available, for straight quotes in verbatim environments
\IfFileExists{upquote.sty}{\usepackage{upquote}}{}
\IfFileExists{microtype.sty}{% use microtype if available
  \usepackage[]{microtype}
  \UseMicrotypeSet[protrusion]{basicmath} % disable protrusion for tt fonts
}{}
\makeatletter
\@ifundefined{KOMAClassName}{% if non-KOMA class
  \IfFileExists{parskip.sty}{%
    \usepackage{parskip}
  }{% else
    \setlength{\parindent}{0pt}
    \setlength{\parskip}{6pt plus 2pt minus 1pt}}
}{% if KOMA class
  \KOMAoptions{parskip=half}}
\makeatother
\usepackage{xcolor}
\IfFileExists{xurl.sty}{\usepackage{xurl}}{} % add URL line breaks if available
\IfFileExists{bookmark.sty}{\usepackage{bookmark}}{\usepackage{hyperref}}
\hypersetup{
  pdftitle={Toolbox},
  pdfauthor={Bhaswar Chakma},
  hidelinks,
  pdfcreator={LaTeX via pandoc}}
\urlstyle{same} % disable monospaced font for URLs
\usepackage{color}
\usepackage{fancyvrb}
\newcommand{\VerbBar}{|}
\newcommand{\VERB}{\Verb[commandchars=\\\{\}]}
\DefineVerbatimEnvironment{Highlighting}{Verbatim}{commandchars=\\\{\}}
% Add ',fontsize=\small' for more characters per line
\usepackage{framed}
\definecolor{shadecolor}{RGB}{248,248,248}
\newenvironment{Shaded}{\begin{snugshade}}{\end{snugshade}}
\newcommand{\AlertTok}[1]{\textcolor[rgb]{0.94,0.16,0.16}{#1}}
\newcommand{\AnnotationTok}[1]{\textcolor[rgb]{0.56,0.35,0.01}{\textbf{\textit{#1}}}}
\newcommand{\AttributeTok}[1]{\textcolor[rgb]{0.77,0.63,0.00}{#1}}
\newcommand{\BaseNTok}[1]{\textcolor[rgb]{0.00,0.00,0.81}{#1}}
\newcommand{\BuiltInTok}[1]{#1}
\newcommand{\CharTok}[1]{\textcolor[rgb]{0.31,0.60,0.02}{#1}}
\newcommand{\CommentTok}[1]{\textcolor[rgb]{0.56,0.35,0.01}{\textit{#1}}}
\newcommand{\CommentVarTok}[1]{\textcolor[rgb]{0.56,0.35,0.01}{\textbf{\textit{#1}}}}
\newcommand{\ConstantTok}[1]{\textcolor[rgb]{0.00,0.00,0.00}{#1}}
\newcommand{\ControlFlowTok}[1]{\textcolor[rgb]{0.13,0.29,0.53}{\textbf{#1}}}
\newcommand{\DataTypeTok}[1]{\textcolor[rgb]{0.13,0.29,0.53}{#1}}
\newcommand{\DecValTok}[1]{\textcolor[rgb]{0.00,0.00,0.81}{#1}}
\newcommand{\DocumentationTok}[1]{\textcolor[rgb]{0.56,0.35,0.01}{\textbf{\textit{#1}}}}
\newcommand{\ErrorTok}[1]{\textcolor[rgb]{0.64,0.00,0.00}{\textbf{#1}}}
\newcommand{\ExtensionTok}[1]{#1}
\newcommand{\FloatTok}[1]{\textcolor[rgb]{0.00,0.00,0.81}{#1}}
\newcommand{\FunctionTok}[1]{\textcolor[rgb]{0.00,0.00,0.00}{#1}}
\newcommand{\ImportTok}[1]{#1}
\newcommand{\InformationTok}[1]{\textcolor[rgb]{0.56,0.35,0.01}{\textbf{\textit{#1}}}}
\newcommand{\KeywordTok}[1]{\textcolor[rgb]{0.13,0.29,0.53}{\textbf{#1}}}
\newcommand{\NormalTok}[1]{#1}
\newcommand{\OperatorTok}[1]{\textcolor[rgb]{0.81,0.36,0.00}{\textbf{#1}}}
\newcommand{\OtherTok}[1]{\textcolor[rgb]{0.56,0.35,0.01}{#1}}
\newcommand{\PreprocessorTok}[1]{\textcolor[rgb]{0.56,0.35,0.01}{\textit{#1}}}
\newcommand{\RegionMarkerTok}[1]{#1}
\newcommand{\SpecialCharTok}[1]{\textcolor[rgb]{0.00,0.00,0.00}{#1}}
\newcommand{\SpecialStringTok}[1]{\textcolor[rgb]{0.31,0.60,0.02}{#1}}
\newcommand{\StringTok}[1]{\textcolor[rgb]{0.31,0.60,0.02}{#1}}
\newcommand{\VariableTok}[1]{\textcolor[rgb]{0.00,0.00,0.00}{#1}}
\newcommand{\VerbatimStringTok}[1]{\textcolor[rgb]{0.31,0.60,0.02}{#1}}
\newcommand{\WarningTok}[1]{\textcolor[rgb]{0.56,0.35,0.01}{\textbf{\textit{#1}}}}
\usepackage{longtable,booktabs,array}
\usepackage{calc} % for calculating minipage widths
% Correct order of tables after \paragraph or \subparagraph
\usepackage{etoolbox}
\makeatletter
\patchcmd\longtable{\par}{\if@noskipsec\mbox{}\fi\par}{}{}
\makeatother
% Allow footnotes in longtable head/foot
\IfFileExists{footnotehyper.sty}{\usepackage{footnotehyper}}{\usepackage{footnote}}
\makesavenoteenv{longtable}
\usepackage{graphicx}
\makeatletter
\def\maxwidth{\ifdim\Gin@nat@width>\linewidth\linewidth\else\Gin@nat@width\fi}
\def\maxheight{\ifdim\Gin@nat@height>\textheight\textheight\else\Gin@nat@height\fi}
\makeatother
% Scale images if necessary, so that they will not overflow the page
% margins by default, and it is still possible to overwrite the defaults
% using explicit options in \includegraphics[width, height, ...]{}
\setkeys{Gin}{width=\maxwidth,height=\maxheight,keepaspectratio}
% Set default figure placement to htbp
\makeatletter
\def\fps@figure{htbp}
\makeatother
\setlength{\emergencystretch}{3em} % prevent overfull lines
\providecommand{\tightlist}{%
  \setlength{\itemsep}{0pt}\setlength{\parskip}{0pt}}
\setcounter{secnumdepth}{5}
\usepackage{booktabs}
\usepackage{amsthm}
\makeatletter
\def\thm@space@setup{%
  \thm@preskip=8pt plus 2pt minus 4pt
  \thm@postskip=\thm@preskip
}
\makeatother
\ifluatex
  \usepackage{selnolig}  % disable illegal ligatures
\fi
\usepackage[]{natbib}
\bibliographystyle{apalike}

\title{Toolbox}
\author{Bhaswar Chakma}
\date{2021-07-08}

\begin{document}
\maketitle

{
\setcounter{tocdepth}{1}
\tableofcontents
}
\hypertarget{section}{%
\chapter*{}\label{section}}
\addcontentsline{toc}{chapter}{}

\hypertarget{python}{%
\chapter{Python}\label{python}}

\hypertarget{pandas}{%
\section{Pandas}\label{pandas}}

NumPy creates ndarrays that must contain values that are of the same data type.
Pandas creates dataframes. Each column in a dataframe is an ndarray. This allows us to have
traditional tables of data where each column can be a different data type.

Important References:

\begin{itemize}
\item
  Series: \url{https://pandas.pydata.org/pandas-docs/stable/reference/series.html}
\item
  DataFrame: \url{https://pandas.pydata.org/pandas-docs/stable/reference/api/pandas.DataFrame.html}
\end{itemize}

\begin{Shaded}
\begin{Highlighting}[]
\ImportTok{import}\NormalTok{ numpy }\ImportTok{as}\NormalTok{ np}
\ImportTok{import}\NormalTok{ pandas }\ImportTok{as}\NormalTok{ pd}
\end{Highlighting}
\end{Shaded}

\hypertarget{series}{%
\subsection{Series}\label{series}}

The basic data structure in pandas is the series. You can construct it in a similar fashion to making a numpy array. The command to make a Series object is
\texttt{pd.Series(data,\ index=index)}. Note that the \texttt{index} argument is optional.

\begin{Shaded}
\begin{Highlighting}[]
\NormalTok{data }\OperatorTok{=}\NormalTok{ pd.Series([}\FloatTok{0.25}\NormalTok{, }\FloatTok{0.5}\NormalTok{, }\FloatTok{0.75}\NormalTok{, }\FloatTok{1.0}\NormalTok{])}
\BuiltInTok{print}\NormalTok{(data)}
\end{Highlighting}
\end{Shaded}

\begin{verbatim}
## 0    0.25
## 1    0.50
## 2    0.75
## 3    1.00
## dtype: float64
\end{verbatim}

\begin{Shaded}
\begin{Highlighting}[]
\BuiltInTok{print}\NormalTok{(}\BuiltInTok{type}\NormalTok{(data)) }\CommentTok{\# data type}
\end{Highlighting}
\end{Shaded}

\begin{verbatim}
## <class 'pandas.core.series.Series'>
\end{verbatim}

\begin{Shaded}
\begin{Highlighting}[]
\BuiltInTok{print}\NormalTok{(data.values) }\CommentTok{\# data values}
\end{Highlighting}
\end{Shaded}

\begin{verbatim}
## [0.25 0.5  0.75 1.  ]
\end{verbatim}

\begin{Shaded}
\begin{Highlighting}[]
\BuiltInTok{print}\NormalTok{(}\BuiltInTok{type}\NormalTok{(data.values)) }\CommentTok{\# The values attribute of the series is a numpy array.}
\end{Highlighting}
\end{Shaded}

\begin{verbatim}
## <class 'numpy.ndarray'>
\end{verbatim}

\begin{Shaded}
\begin{Highlighting}[]
\BuiltInTok{print}\NormalTok{(data.index) }
\end{Highlighting}
\end{Shaded}

\begin{verbatim}
## RangeIndex(start=0, stop=4, step=1)
\end{verbatim}

\begin{Shaded}
\begin{Highlighting}[]
\BuiltInTok{print}\NormalTok{(}\BuiltInTok{type}\NormalTok{(data.index)) }\CommentTok{\# the row names are known as the index}
\end{Highlighting}
\end{Shaded}

\begin{verbatim}
## <class 'pandas.core.indexes.range.RangeIndex'>
\end{verbatim}

You can subset a pandas series like other python objects

\begin{Shaded}
\begin{Highlighting}[]
\BuiltInTok{print}\NormalTok{(data) }\CommentTok{\# example data}
\end{Highlighting}
\end{Shaded}

\begin{verbatim}
## 0    0.25
## 1    0.50
## 2    0.75
## 3    1.00
## dtype: float64
\end{verbatim}

\begin{Shaded}
\begin{Highlighting}[]
\BuiltInTok{print}\NormalTok{(data[}\DecValTok{1}\NormalTok{]) }\CommentTok{\# select the 2nd value}
\end{Highlighting}
\end{Shaded}

\begin{verbatim}
## 0.5
\end{verbatim}

\begin{Shaded}
\begin{Highlighting}[]
\BuiltInTok{print}\NormalTok{(}\BuiltInTok{type}\NormalTok{(data[}\DecValTok{1}\NormalTok{])) }\CommentTok{\# when you select only one value, it simplifies the object}
\end{Highlighting}
\end{Shaded}

\begin{verbatim}
## <class 'numpy.float64'>
\end{verbatim}

\begin{Shaded}
\begin{Highlighting}[]
\BuiltInTok{print}\NormalTok{(data[}\DecValTok{1}\NormalTok{:}\DecValTok{3}\NormalTok{])}
\end{Highlighting}
\end{Shaded}

\begin{verbatim}
## 1    0.50
## 2    0.75
## dtype: float64
\end{verbatim}

\begin{Shaded}
\begin{Highlighting}[]
\BuiltInTok{print}\NormalTok{(}\BuiltInTok{type}\NormalTok{(data[}\DecValTok{1}\NormalTok{:}\DecValTok{3}\NormalTok{])) }\CommentTok{\# slicing / selecting multiple values returns a series}
\end{Highlighting}
\end{Shaded}

\begin{verbatim}
## <class 'pandas.core.series.Series'>
\end{verbatim}

You can also do fancy indexing by subsetting w/a numpy array e.g.~repeat observations.

\begin{Shaded}
\begin{Highlighting}[]
\BuiltInTok{print}\NormalTok{(data[np.array([}\DecValTok{1}\NormalTok{, }\DecValTok{0}\NormalTok{, }\DecValTok{1}\NormalTok{, }\DecValTok{2}\NormalTok{])])}
\end{Highlighting}
\end{Shaded}

\begin{verbatim}
## 1    0.50
## 0    0.25
## 1    0.50
## 2    0.75
## dtype: float64
\end{verbatim}

Pandas uses a 0-based index by default. You may also specify the index values.

\begin{Shaded}
\begin{Highlighting}[]
\NormalTok{data }\OperatorTok{=}\NormalTok{ pd.Series([}\FloatTok{0.25}\NormalTok{, }\FloatTok{0.5}\NormalTok{, }\FloatTok{0.75}\NormalTok{, }\FloatTok{1.0}\NormalTok{],}
\NormalTok{index }\OperatorTok{=}\NormalTok{ [}\StringTok{\textquotesingle{}a\textquotesingle{}}\NormalTok{, }\StringTok{\textquotesingle{}b\textquotesingle{}}\NormalTok{, }\StringTok{\textquotesingle{}c\textquotesingle{}}\NormalTok{, }\StringTok{\textquotesingle{}d\textquotesingle{}}\NormalTok{])}
\BuiltInTok{print}\NormalTok{(data)}
\end{Highlighting}
\end{Shaded}

\begin{verbatim}
## a    0.25
## b    0.50
## c    0.75
## d    1.00
## dtype: float64
\end{verbatim}

\begin{Shaded}
\begin{Highlighting}[]
\NormalTok{data.values}
\end{Highlighting}
\end{Shaded}

\begin{verbatim}
## array([0.25, 0.5 , 0.75, 1.  ])
\end{verbatim}

\begin{Shaded}
\begin{Highlighting}[]
\NormalTok{data.index}
\end{Highlighting}
\end{Shaded}

\begin{verbatim}
## Index(['a', 'b', 'c', 'd'], dtype='object')
\end{verbatim}

\begin{itemize}
\tightlist
\item
  subset with index position
\end{itemize}

\begin{Shaded}
\begin{Highlighting}[]
\NormalTok{data[}\DecValTok{1}\NormalTok{]}
\end{Highlighting}
\end{Shaded}

\begin{verbatim}
## 0.5
\end{verbatim}

\begin{itemize}
\tightlist
\item
  subset with index name
\end{itemize}

\begin{Shaded}
\begin{Highlighting}[]
\NormalTok{data[}\StringTok{"a"}\NormalTok{]}
\end{Highlighting}
\end{Shaded}

\begin{verbatim}
## 0.25
\end{verbatim}

\hypertarget{r}{%
\chapter{R}\label{r}}

\hypertarget{sql}{%
\chapter{SQL}\label{sql}}

Coming!

  \bibliography{book.bib,packages.bib}

\end{document}
