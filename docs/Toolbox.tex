% Options for packages loaded elsewhere
\PassOptionsToPackage{unicode}{hyperref}
\PassOptionsToPackage{hyphens}{url}
%
\documentclass[
]{book}
\usepackage{amsmath,amssymb}
\usepackage{lmodern}
\usepackage{ifxetex,ifluatex}
\ifnum 0\ifxetex 1\fi\ifluatex 1\fi=0 % if pdftex
  \usepackage[T1]{fontenc}
  \usepackage[utf8]{inputenc}
  \usepackage{textcomp} % provide euro and other symbols
\else % if luatex or xetex
  \usepackage{unicode-math}
  \defaultfontfeatures{Scale=MatchLowercase}
  \defaultfontfeatures[\rmfamily]{Ligatures=TeX,Scale=1}
\fi
% Use upquote if available, for straight quotes in verbatim environments
\IfFileExists{upquote.sty}{\usepackage{upquote}}{}
\IfFileExists{microtype.sty}{% use microtype if available
  \usepackage[]{microtype}
  \UseMicrotypeSet[protrusion]{basicmath} % disable protrusion for tt fonts
}{}
\makeatletter
\@ifundefined{KOMAClassName}{% if non-KOMA class
  \IfFileExists{parskip.sty}{%
    \usepackage{parskip}
  }{% else
    \setlength{\parindent}{0pt}
    \setlength{\parskip}{6pt plus 2pt minus 1pt}}
}{% if KOMA class
  \KOMAoptions{parskip=half}}
\makeatother
\usepackage{xcolor}
\IfFileExists{xurl.sty}{\usepackage{xurl}}{} % add URL line breaks if available
\IfFileExists{bookmark.sty}{\usepackage{bookmark}}{\usepackage{hyperref}}
\hypersetup{
  pdftitle={Toolbox},
  pdfauthor={Bhaswar Chakma},
  hidelinks,
  pdfcreator={LaTeX via pandoc}}
\urlstyle{same} % disable monospaced font for URLs
\usepackage{color}
\usepackage{fancyvrb}
\newcommand{\VerbBar}{|}
\newcommand{\VERB}{\Verb[commandchars=\\\{\}]}
\DefineVerbatimEnvironment{Highlighting}{Verbatim}{commandchars=\\\{\}}
% Add ',fontsize=\small' for more characters per line
\usepackage{framed}
\definecolor{shadecolor}{RGB}{248,248,248}
\newenvironment{Shaded}{\begin{snugshade}}{\end{snugshade}}
\newcommand{\AlertTok}[1]{\textcolor[rgb]{0.94,0.16,0.16}{#1}}
\newcommand{\AnnotationTok}[1]{\textcolor[rgb]{0.56,0.35,0.01}{\textbf{\textit{#1}}}}
\newcommand{\AttributeTok}[1]{\textcolor[rgb]{0.77,0.63,0.00}{#1}}
\newcommand{\BaseNTok}[1]{\textcolor[rgb]{0.00,0.00,0.81}{#1}}
\newcommand{\BuiltInTok}[1]{#1}
\newcommand{\CharTok}[1]{\textcolor[rgb]{0.31,0.60,0.02}{#1}}
\newcommand{\CommentTok}[1]{\textcolor[rgb]{0.56,0.35,0.01}{\textit{#1}}}
\newcommand{\CommentVarTok}[1]{\textcolor[rgb]{0.56,0.35,0.01}{\textbf{\textit{#1}}}}
\newcommand{\ConstantTok}[1]{\textcolor[rgb]{0.00,0.00,0.00}{#1}}
\newcommand{\ControlFlowTok}[1]{\textcolor[rgb]{0.13,0.29,0.53}{\textbf{#1}}}
\newcommand{\DataTypeTok}[1]{\textcolor[rgb]{0.13,0.29,0.53}{#1}}
\newcommand{\DecValTok}[1]{\textcolor[rgb]{0.00,0.00,0.81}{#1}}
\newcommand{\DocumentationTok}[1]{\textcolor[rgb]{0.56,0.35,0.01}{\textbf{\textit{#1}}}}
\newcommand{\ErrorTok}[1]{\textcolor[rgb]{0.64,0.00,0.00}{\textbf{#1}}}
\newcommand{\ExtensionTok}[1]{#1}
\newcommand{\FloatTok}[1]{\textcolor[rgb]{0.00,0.00,0.81}{#1}}
\newcommand{\FunctionTok}[1]{\textcolor[rgb]{0.00,0.00,0.00}{#1}}
\newcommand{\ImportTok}[1]{#1}
\newcommand{\InformationTok}[1]{\textcolor[rgb]{0.56,0.35,0.01}{\textbf{\textit{#1}}}}
\newcommand{\KeywordTok}[1]{\textcolor[rgb]{0.13,0.29,0.53}{\textbf{#1}}}
\newcommand{\NormalTok}[1]{#1}
\newcommand{\OperatorTok}[1]{\textcolor[rgb]{0.81,0.36,0.00}{\textbf{#1}}}
\newcommand{\OtherTok}[1]{\textcolor[rgb]{0.56,0.35,0.01}{#1}}
\newcommand{\PreprocessorTok}[1]{\textcolor[rgb]{0.56,0.35,0.01}{\textit{#1}}}
\newcommand{\RegionMarkerTok}[1]{#1}
\newcommand{\SpecialCharTok}[1]{\textcolor[rgb]{0.00,0.00,0.00}{#1}}
\newcommand{\SpecialStringTok}[1]{\textcolor[rgb]{0.31,0.60,0.02}{#1}}
\newcommand{\StringTok}[1]{\textcolor[rgb]{0.31,0.60,0.02}{#1}}
\newcommand{\VariableTok}[1]{\textcolor[rgb]{0.00,0.00,0.00}{#1}}
\newcommand{\VerbatimStringTok}[1]{\textcolor[rgb]{0.31,0.60,0.02}{#1}}
\newcommand{\WarningTok}[1]{\textcolor[rgb]{0.56,0.35,0.01}{\textbf{\textit{#1}}}}
\usepackage{longtable,booktabs,array}
\usepackage{calc} % for calculating minipage widths
% Correct order of tables after \paragraph or \subparagraph
\usepackage{etoolbox}
\makeatletter
\patchcmd\longtable{\par}{\if@noskipsec\mbox{}\fi\par}{}{}
\makeatother
% Allow footnotes in longtable head/foot
\IfFileExists{footnotehyper.sty}{\usepackage{footnotehyper}}{\usepackage{footnote}}
\makesavenoteenv{longtable}
\usepackage{graphicx}
\makeatletter
\def\maxwidth{\ifdim\Gin@nat@width>\linewidth\linewidth\else\Gin@nat@width\fi}
\def\maxheight{\ifdim\Gin@nat@height>\textheight\textheight\else\Gin@nat@height\fi}
\makeatother
% Scale images if necessary, so that they will not overflow the page
% margins by default, and it is still possible to overwrite the defaults
% using explicit options in \includegraphics[width, height, ...]{}
\setkeys{Gin}{width=\maxwidth,height=\maxheight,keepaspectratio}
% Set default figure placement to htbp
\makeatletter
\def\fps@figure{htbp}
\makeatother
\setlength{\emergencystretch}{3em} % prevent overfull lines
\providecommand{\tightlist}{%
  \setlength{\itemsep}{0pt}\setlength{\parskip}{0pt}}
\setcounter{secnumdepth}{5}
\usepackage{booktabs}
\usepackage{amsthm}
\makeatletter
\def\thm@space@setup{%
  \thm@preskip=8pt plus 2pt minus 4pt
  \thm@postskip=\thm@preskip
}
\makeatother
\ifluatex
  \usepackage{selnolig}  % disable illegal ligatures
\fi
\usepackage[]{natbib}
\bibliographystyle{apalike}

\title{Toolbox}
\author{Bhaswar Chakma}
\date{2021-07-17}

\begin{document}
\maketitle

{
\setcounter{tocdepth}{1}
\tableofcontents
}
\hypertarget{section}{%
\chapter*{}\label{section}}
\addcontentsline{toc}{chapter}{}

\hypertarget{dplyr-vs-pandas}{%
\chapter{dplyr vs pandas}\label{dplyr-vs-pandas}}

We will use the five \href{https://dplyr.tidyverse.org/}{dplyr verbs} (also \href{https://pandas.pydata.org/pandas-docs/stable/getting_started/comparison/comparison_with_r.html}{pandas' guide} is also helpful) for comparison

\begin{itemize}
\item
  \texttt{select()} picks variables based on their names.
\item
  \texttt{mutate()} adds new variables that are functions of existing variables
\item
  \texttt{filter()} picks cases based on their values.
\item
  \texttt{summarise()} reduces multiple values down to a single summary.
\item
  \texttt{arrange()} changes the ordering of the rows.
\end{itemize}

and use the following toy data to apply the verbs.

\begin{tabular}{l|l|r}
\hline
name & gender & grade\\
\hline
Barney & Male & 10\\
\hline
Ted & Male & 11\\
\hline
Marshall & Male & 13\\
\hline
Lilly & Female & 12\\
\hline
Robin & Female & 14\\
\hline
\end{tabular}

{\textbf{Create Toy Data}}

dplyr

pandas

\begin{Shaded}
\begin{Highlighting}[]
\NormalTok{df }\OtherTok{\textless{}{-}} \FunctionTok{tibble}\NormalTok{(}
  \AttributeTok{name =} \FunctionTok{c}\NormalTok{(}\StringTok{"Barney"}\NormalTok{, }\StringTok{"Ted"}\NormalTok{, }\StringTok{"Marshall"}\NormalTok{,}
           \StringTok{"Lilly"}\NormalTok{,}\StringTok{"Robin"}\NormalTok{),}
  \AttributeTok{gender =} \FunctionTok{c}\NormalTok{(}\StringTok{"Male"}\NormalTok{, }\StringTok{"Male"}\NormalTok{,}\StringTok{"Male"}\NormalTok{,}
             \StringTok{"Female"}\NormalTok{, }\StringTok{"Female"}\NormalTok{),}
  \AttributeTok{grade =} \FunctionTok{c}\NormalTok{(}\DecValTok{10}\NormalTok{, }\DecValTok{11}\NormalTok{, }\DecValTok{13}\NormalTok{, }\DecValTok{12}\NormalTok{, }\DecValTok{14}\NormalTok{)}
\NormalTok{)}
\NormalTok{df}
\end{Highlighting}
\end{Shaded}

\begin{verbatim}
## # A tibble: 5 x 3
##   name     gender grade
##   <chr>    <chr>  <dbl>
## 1 Barney   Male      10
## 2 Ted      Male      11
## 3 Marshall Male      13
## 4 Lilly    Female    12
## 5 Robin    Female    14
\end{verbatim}

\begin{Shaded}
\begin{Highlighting}[]
\NormalTok{df }\OperatorTok{=}\NormalTok{ pd.DataFrame(\{}
  \StringTok{\textquotesingle{}name\textquotesingle{}}\NormalTok{:[}\StringTok{"Barney"}\NormalTok{, }\StringTok{"Ted"}\NormalTok{, }\StringTok{"Marshall"}\NormalTok{,}
          \StringTok{"Lilly"}\NormalTok{, }\StringTok{"Robin"}\NormalTok{],}
  \StringTok{\textquotesingle{}gender\textquotesingle{}}\NormalTok{:[}\StringTok{"Male"}\NormalTok{, }\StringTok{"Male"}\NormalTok{,}\StringTok{"Male"}\NormalTok{, }
            \StringTok{"Female"}\NormalTok{, }\StringTok{"Female"}\NormalTok{],}
  \StringTok{\textquotesingle{}grade\textquotesingle{}}\NormalTok{:[}\DecValTok{10}\NormalTok{, }\DecValTok{11}\NormalTok{, }\DecValTok{13}\NormalTok{, }\DecValTok{12}\NormalTok{, }\DecValTok{14}\NormalTok{] }
\NormalTok{\})}
\NormalTok{df}
\end{Highlighting}
\end{Shaded}

\begin{verbatim}
##        name  gender  grade
## 0    Barney    Male     10
## 1       Ted    Male     11
## 2  Marshall    Male     13
## 3     Lilly  Female     12
## 4     Robin  Female     14
\end{verbatim}

{\textbf{Check Data Structure}}

dplyr

pandas

\begin{Shaded}
\begin{Highlighting}[]
\FunctionTok{glimpse}\NormalTok{(df)}
\end{Highlighting}
\end{Shaded}

\begin{verbatim}
## Rows: 5
## Columns: 3
## $ name   <chr> "Barney", "Ted", "Marshall", "Lilly", "Robin"
## $ gender <chr> "Male", "Male", "Male", "Female", "Female"
## $ grade  <dbl> 10, 11, 13, 12, 14
\end{verbatim}

\begin{Shaded}
\begin{Highlighting}[]
\NormalTok{df.dtypes}
\end{Highlighting}
\end{Shaded}

\begin{verbatim}
## name      object
## gender    object
## grade      int64
## dtype: object
\end{verbatim}

\begin{Shaded}
\begin{Highlighting}[]
\NormalTok{df.shape}
\end{Highlighting}
\end{Shaded}

\begin{verbatim}
## (5, 3)
\end{verbatim}

\begin{Shaded}
\begin{Highlighting}[]
\NormalTok{df.info()}
\end{Highlighting}
\end{Shaded}

\begin{verbatim}
## <class 'pandas.core.frame.DataFrame'>
## RangeIndex: 5 entries, 0 to 4
## Data columns (total 3 columns):
## name      5 non-null object
## gender    5 non-null object
## grade     5 non-null int64
## dtypes: int64(1), object(2)
## memory usage: 248.0+ bytes
\end{verbatim}

\hypertarget{select}{%
\section{select()}\label{select}}

{\textbf{Task: Pick the variables \texttt{name} and \texttt{grade}.
}}

dplyr

pandas

\begin{Shaded}
\begin{Highlighting}[]
\NormalTok{df }\SpecialCharTok{\%\textgreater{}\%} 
  \FunctionTok{select}\NormalTok{(name, grade)}
\end{Highlighting}
\end{Shaded}

\begin{verbatim}
## # A tibble: 5 x 2
##   name     grade
##   <chr>    <dbl>
## 1 Barney      10
## 2 Ted         11
## 3 Marshall    13
## 4 Lilly       12
## 5 Robin       14
\end{verbatim}

\begin{Shaded}
\begin{Highlighting}[]
\NormalTok{df[[}\StringTok{\textquotesingle{}name\textquotesingle{}}\NormalTok{, }\StringTok{\textquotesingle{}grade\textquotesingle{}}\NormalTok{]]}
\end{Highlighting}
\end{Shaded}

\begin{verbatim}
##        name  grade
## 0    Barney     10
## 1       Ted     11
## 2  Marshall     13
## 3     Lilly     12
## 4     Robin     14
\end{verbatim}

\begin{Shaded}
\begin{Highlighting}[]
\CommentTok{\# or}
\NormalTok{df.drop(columns }\OperatorTok{=}\NormalTok{ [}\StringTok{\textquotesingle{}grade\textquotesingle{}}\NormalTok{])}
\end{Highlighting}
\end{Shaded}

\begin{verbatim}
##        name  gender
## 0    Barney    Male
## 1       Ted    Male
## 2  Marshall    Male
## 3     Lilly  Female
## 4     Robin  Female
\end{verbatim}

\begin{Shaded}
\begin{Highlighting}[]
\CommentTok{\# or}
\NormalTok{df.drop([}\StringTok{\textquotesingle{}grade\textquotesingle{}}\NormalTok{], axis }\OperatorTok{=} \DecValTok{1}\NormalTok{)}
\end{Highlighting}
\end{Shaded}

\begin{verbatim}
##        name  gender
## 0    Barney    Male
## 1       Ted    Male
## 2  Marshall    Male
## 3     Lilly  Female
## 4     Robin  Female
\end{verbatim}

\hypertarget{mutate}{%
\section{mutate()}\label{mutate}}

{\textbf{Task: Generate a variable \texttt{grade\_p}, expressing grade out of 100.
}}

dplyr

pandas

\begin{Shaded}
\begin{Highlighting}[]
\NormalTok{df }\SpecialCharTok{\%\textgreater{}\%} 
  \FunctionTok{mutate}\NormalTok{(}\AttributeTok{grade\_p =}\NormalTok{ grade}\SpecialCharTok{/}\DecValTok{20}\SpecialCharTok{*}\DecValTok{100}\NormalTok{)}
\end{Highlighting}
\end{Shaded}

\begin{verbatim}
## # A tibble: 5 x 4
##   name     gender grade grade_p
##   <chr>    <chr>  <dbl>   <dbl>
## 1 Barney   Male      10      50
## 2 Ted      Male      11      55
## 3 Marshall Male      13      65
## 4 Lilly    Female    12      60
## 5 Robin    Female    14      70
\end{verbatim}

\begin{Shaded}
\begin{Highlighting}[]
\NormalTok{df[}\StringTok{\textquotesingle{}grade\_p\textquotesingle{}}\NormalTok{] }\OperatorTok{=}\NormalTok{ df[}\StringTok{\textquotesingle{}grade\textquotesingle{}}\NormalTok{]}\OperatorTok{/}\DecValTok{20}\OperatorTok{*}\DecValTok{100}
\NormalTok{df}
\end{Highlighting}
\end{Shaded}

\begin{verbatim}
##        name  gender  grade  grade_p
## 0    Barney    Male     10     50.0
## 1       Ted    Male     11     55.0
## 2  Marshall    Male     13     65.0
## 3     Lilly  Female     12     60.0
## 4     Robin  Female     14     70.0
\end{verbatim}

\begin{Shaded}
\begin{Highlighting}[]
\CommentTok{\# now drop the newly created variable}
\NormalTok{df.drop(columns }\OperatorTok{=} \StringTok{\textquotesingle{}grade\_p\textquotesingle{}}\NormalTok{, inplace }\OperatorTok{=} \VariableTok{True}\NormalTok{)}
\end{Highlighting}
\end{Shaded}

\hypertarget{filter}{%
\section{filter()}\label{filter}}

{\textbf{Task: Keep Barney or females.
}}

dplyr

pandas

\begin{Shaded}
\begin{Highlighting}[]
\NormalTok{df }\SpecialCharTok{\%\textgreater{}\%} 
  \FunctionTok{filter}\NormalTok{(name }\SpecialCharTok{==} \StringTok{"Barney"}\SpecialCharTok{|}
\NormalTok{         gender }\SpecialCharTok{==} \StringTok{"Female"}\NormalTok{)}
\end{Highlighting}
\end{Shaded}

\begin{verbatim}
## # A tibble: 3 x 3
##   name   gender grade
##   <chr>  <chr>  <dbl>
## 1 Barney Male      10
## 2 Lilly  Female    12
## 3 Robin  Female    14
\end{verbatim}

\begin{Shaded}
\begin{Highlighting}[]
\CommentTok{\# similar to base R}
\NormalTok{df[(df[}\StringTok{"name"}\NormalTok{] }\OperatorTok{==} \StringTok{"Barney"}\NormalTok{) }\OperatorTok{|} 
\NormalTok{   (df[}\StringTok{"gender"}\NormalTok{] }\OperatorTok{==} \StringTok{"Female"}\NormalTok{)]}
\end{Highlighting}
\end{Shaded}

\begin{verbatim}
##      name  gender  grade
## 0  Barney    Male     10
## 3   Lilly  Female     12
## 4   Robin  Female     14
\end{verbatim}

\begin{Shaded}
\begin{Highlighting}[]
\CommentTok{\# query with \textquotesingle{}\textquotesingle{}; need to use "" for conditions}
\NormalTok{df.query(}\StringTok{\textquotesingle{}name == "Barney"|gender == "Female"\textquotesingle{}}\NormalTok{)}
\end{Highlighting}
\end{Shaded}

\begin{verbatim}
##      name  gender  grade
## 0  Barney    Male     10
## 3   Lilly  Female     12
## 4   Robin  Female     14
\end{verbatim}

\begin{Shaded}
\begin{Highlighting}[]
\CommentTok{\# query with ""; need to use \textquotesingle{}\textquotesingle{} for conditions}
\NormalTok{df.query(}\StringTok{"name == \textquotesingle{}Barney\textquotesingle{}| gender == \textquotesingle{}Female\textquotesingle{}"}\NormalTok{)}
\end{Highlighting}
\end{Shaded}

\begin{verbatim}
##      name  gender  grade
## 0  Barney    Male     10
## 3   Lilly  Female     12
## 4   Robin  Female     14
\end{verbatim}

\hypertarget{group_by-and-summarize}{%
\section{group\_by() and summarize()}\label{group_by-and-summarize}}

{\textbf{Task: Grouped by gender, find mean grade.
}}

dplyr

pandas

\begin{Shaded}
\begin{Highlighting}[]
\NormalTok{df }\SpecialCharTok{\%\textgreater{}\%} 
  \FunctionTok{group\_by}\NormalTok{(gender) }\SpecialCharTok{\%\textgreater{}\%} 
  \FunctionTok{summarize}\NormalTok{(}\AttributeTok{avg\_grade =} \FunctionTok{mean}\NormalTok{(grade))}
\end{Highlighting}
\end{Shaded}

\begin{verbatim}
## # A tibble: 2 x 2
##   gender avg_grade
##   <chr>      <dbl>
## 1 Female      13  
## 2 Male        11.3
\end{verbatim}

\begin{Shaded}
\begin{Highlighting}[]
\CommentTok{\# returns a series}
\NormalTok{df.groupby(}\StringTok{"gender"}\NormalTok{)[}\StringTok{\textquotesingle{}grade\textquotesingle{}}\NormalTok{].mean()}
\end{Highlighting}
\end{Shaded}

\begin{verbatim}
## gender
## Female    13.000000
## Male      11.333333
## Name: grade, dtype: float64
\end{verbatim}

\begin{Shaded}
\begin{Highlighting}[]
\CommentTok{\# returns a data frame}
\NormalTok{df[[}\StringTok{\textquotesingle{}gender\textquotesingle{}}\NormalTok{, }\StringTok{\textquotesingle{}grade\textquotesingle{}}\NormalTok{]].groupby(}\StringTok{"gender"}\NormalTok{).mean()}
\end{Highlighting}
\end{Shaded}

\begin{verbatim}
##             grade
## gender           
## Female  13.000000
## Male    11.333333
\end{verbatim}

{\textbf{Task: Grouped by gender, find mean, median, minimum, and maximum grade.
}}

dplyr

pandas

\begin{Shaded}
\begin{Highlighting}[]
\NormalTok{df }\SpecialCharTok{\%\textgreater{}\%} 
  \FunctionTok{group\_by}\NormalTok{(gender) }\SpecialCharTok{\%\textgreater{}\%} 
  \FunctionTok{summarize}\NormalTok{(}\AttributeTok{mean =} \FunctionTok{mean}\NormalTok{(grade),}
            \AttributeTok{median =} \FunctionTok{median}\NormalTok{(grade),}
            \AttributeTok{min =} \FunctionTok{min}\NormalTok{(grade),}
            \AttributeTok{max =} \FunctionTok{max}\NormalTok{(grade))}
\end{Highlighting}
\end{Shaded}

\begin{verbatim}
## # A tibble: 2 x 5
##   gender  mean median   min   max
##   <chr>  <dbl>  <dbl> <dbl> <dbl>
## 1 Female  13       13    12    14
## 2 Male    11.3     11    10    13
\end{verbatim}

\begin{Shaded}
\begin{Highlighting}[]
\NormalTok{df.groupby(}\StringTok{"gender"}\NormalTok{)[}\StringTok{\textquotesingle{}grade\textquotesingle{}}\NormalTok{].agg(}
  \CommentTok{\# provide a dictionary}
\NormalTok{  \{}\CommentTok{\# variable name followed by function}
    \StringTok{\textquotesingle{}mean\textquotesingle{}}\NormalTok{: }\StringTok{\textquotesingle{}mean\textquotesingle{}}\NormalTok{,}
    \StringTok{\textquotesingle{}median\textquotesingle{}}\NormalTok{: }\StringTok{\textquotesingle{}median\textquotesingle{}}\NormalTok{,}
    \StringTok{\textquotesingle{}min\textquotesingle{}}\NormalTok{: }\StringTok{\textquotesingle{}min\textquotesingle{}}\NormalTok{,}
    \StringTok{\textquotesingle{}max\textquotesingle{}}\NormalTok{: }\StringTok{\textquotesingle{}max\textquotesingle{}}
\NormalTok{  \}}
\NormalTok{)}
\end{Highlighting}
\end{Shaded}

\begin{verbatim}
##              mean  median  min  max
## gender                             
## Female  13.000000      13   12   14
## Male    11.333333      11   10   13
## 
## C:/Users/Bhaswar/AppData/Local/r-miniconda/envs/r-reticulate/python.exe:7: FutureWarning: using a dict on a Series for aggregation
## is deprecated and will be removed in a future version. Use                 named aggregation instead.
## 
##     >>> grouper.agg(name_1=func_1, name_2=func_2)
\end{verbatim}

\hypertarget{arrange}{%
\section{arrange()}\label{arrange}}

{\textbf{Task: Arrange grade in ascending order.
}}

dplyr

pandas

\begin{Shaded}
\begin{Highlighting}[]
\NormalTok{df }\SpecialCharTok{\%\textgreater{}\%} 
  \FunctionTok{arrange}\NormalTok{(grade)}
\end{Highlighting}
\end{Shaded}

\begin{verbatim}
## # A tibble: 5 x 3
##   name     gender grade
##   <chr>    <chr>  <dbl>
## 1 Barney   Male      10
## 2 Ted      Male      11
## 3 Lilly    Female    12
## 4 Marshall Male      13
## 5 Robin    Female    14
\end{verbatim}

\begin{Shaded}
\begin{Highlighting}[]
\NormalTok{df.sort\_values(}\StringTok{\textquotesingle{}grade\textquotesingle{}}\NormalTok{)}
\end{Highlighting}
\end{Shaded}

\begin{verbatim}
##        name  gender  grade
## 0    Barney    Male     10
## 1       Ted    Male     11
## 3     Lilly  Female     12
## 2  Marshall    Male     13
## 4     Robin  Female     14
\end{verbatim}

{\textbf{Task: Arrange grade in ascending order.
}}

dplyr

pandas

\begin{Shaded}
\begin{Highlighting}[]
\NormalTok{df }\SpecialCharTok{\%\textgreater{}\%} \FunctionTok{arrange}\NormalTok{(}\FunctionTok{desc}\NormalTok{(grade))}
\end{Highlighting}
\end{Shaded}

\begin{verbatim}
## # A tibble: 5 x 3
##   name     gender grade
##   <chr>    <chr>  <dbl>
## 1 Robin    Female    14
## 2 Marshall Male      13
## 3 Lilly    Female    12
## 4 Ted      Male      11
## 5 Barney   Male      10
\end{verbatim}

\begin{Shaded}
\begin{Highlighting}[]
\NormalTok{df.sort\_values(}\StringTok{\textquotesingle{}grade\textquotesingle{}}\NormalTok{, ascending }\OperatorTok{=} \VariableTok{False}\NormalTok{)}
\end{Highlighting}
\end{Shaded}

\begin{verbatim}
##        name  gender  grade
## 4     Robin  Female     14
## 2  Marshall    Male     13
## 3     Lilly  Female     12
## 1       Ted    Male     11
## 0    Barney    Male     10
\end{verbatim}

{\textbf{Task: Arrange gender in ascending order then arrange grade in descending order.
}}

dplyr

pandas

\begin{Shaded}
\begin{Highlighting}[]
\NormalTok{df }\SpecialCharTok{\%\textgreater{}\%}
  \FunctionTok{arrange}\NormalTok{(gender, }\FunctionTok{desc}\NormalTok{(grade))}
\end{Highlighting}
\end{Shaded}

\begin{verbatim}
## # A tibble: 5 x 3
##   name     gender grade
##   <chr>    <chr>  <dbl>
## 1 Robin    Female    14
## 2 Lilly    Female    12
## 3 Marshall Male      13
## 4 Ted      Male      11
## 5 Barney   Male      10
\end{verbatim}

\begin{Shaded}
\begin{Highlighting}[]
\NormalTok{df.sort\_values([}\StringTok{\textquotesingle{}gender\textquotesingle{}}\NormalTok{,}\StringTok{\textquotesingle{}grade\textquotesingle{}}\NormalTok{],}
\NormalTok{                ascending }\OperatorTok{=}\NormalTok{ [}\VariableTok{True}\NormalTok{, }\VariableTok{False}\NormalTok{])}
\end{Highlighting}
\end{Shaded}

\begin{verbatim}
##        name  gender  grade
## 4     Robin  Female     14
## 3     Lilly  Female     12
## 2  Marshall    Male     13
## 1       Ted    Male     11
## 0    Barney    Male     10
\end{verbatim}

\hypertarget{base-python}{%
\chapter{Base Python}\label{base-python}}

\hypertarget{map}{%
\section{\texorpdfstring{\texttt{map()}}{map()}}\label{map}}

\texttt{map()} lets you apply a function to each element of a list.

\begin{Shaded}
\begin{Highlighting}[]
\CommentTok{\# toy list}
\NormalTok{toy\_list }\OperatorTok{=}\NormalTok{ [}\DecValTok{1}\NormalTok{, }\DecValTok{200}\NormalTok{, }\DecValTok{3}\NormalTok{, }\DecValTok{400}\NormalTok{]}

\CommentTok{\# Create toy function}
\KeywordTok{def}\NormalTok{ smaller\_than\_100(k):}
  \ControlFlowTok{if}\NormalTok{ k }\OperatorTok{\textless{}} \DecValTok{100}\NormalTok{:}
    \ControlFlowTok{return} \VariableTok{True}
  \ControlFlowTok{else}\NormalTok{:}
    \VariableTok{False}
\CommentTok{\# test the function}
\NormalTok{smaller\_than\_100(}\DecValTok{2}\NormalTok{)}
\end{Highlighting}
\end{Shaded}

\begin{verbatim}
## True
\end{verbatim}

\begin{Shaded}
\begin{Highlighting}[]
\CommentTok{\# apply it to toy\_list}
\NormalTok{mapped }\OperatorTok{=} \BuiltInTok{map}\NormalTok{(smaller\_than\_100, toy\_list)}
\BuiltInTok{print}\NormalTok{(mapped) }\CommentTok{\# doesn\textquotesingle{}t provide the desired output; use loop}
\end{Highlighting}
\end{Shaded}

\begin{verbatim}
## <map object at 0x000000003082B5F8>
\end{verbatim}

\begin{Shaded}
\begin{Highlighting}[]
\ControlFlowTok{for}\NormalTok{ i }\KeywordTok{in}\NormalTok{ mapped:}
    \BuiltInTok{print}\NormalTok{(i)}
\end{Highlighting}
\end{Shaded}

\begin{verbatim}
## True
## None
## True
## None
\end{verbatim}

\hypertarget{r-strings}{%
\chapter{R Strings}\label{r-strings}}

\hypertarget{string-manupulation-with-base-r-functions}{%
\section{String Manupulation with Base R Functions}\label{string-manupulation-with-base-r-functions}}

There are many functions in base R for basic string manipulation.

Function

Example

{\textbf{\texttt{nchar()}
}}

\begin{Shaded}
\begin{Highlighting}[]
\NormalTok{y }\OtherTok{\textless{}{-}} \FunctionTok{c}\NormalTok{(}\StringTok{"Hello"}\NormalTok{, }\StringTok{"World"}\NormalTok{, }\StringTok{"Hello"}\NormalTok{, }\StringTok{"Universe"}\NormalTok{)}
\FunctionTok{nchar}\NormalTok{(y) }\CommentTok{\# Returns number of characters}
\end{Highlighting}
\end{Shaded}

\begin{verbatim}
## [1] 5 5 5 8
\end{verbatim}

{\textbf{\texttt{tolower()}
}}

\begin{Shaded}
\begin{Highlighting}[]
\FunctionTok{tolower}\NormalTok{(y)}
\end{Highlighting}
\end{Shaded}

\begin{verbatim}
## [1] "hello"    "world"    "hello"    "universe"
\end{verbatim}

{\textbf{\texttt{toupper()}
}}

\begin{Shaded}
\begin{Highlighting}[]
\FunctionTok{toupper}\NormalTok{(y)}
\end{Highlighting}
\end{Shaded}

\begin{verbatim}
## [1] "HELLO"    "WORLD"    "HELLO"    "UNIVERSE"
\end{verbatim}

{\textbf{\texttt{chartr()}
}}

\begin{Shaded}
\begin{Highlighting}[]
\FunctionTok{chartr}\NormalTok{(}\StringTok{"oe"}\NormalTok{, }\StringTok{"$\#"}\NormalTok{, y)}\CommentTok{\#o becomes $; e becomes \#}
\end{Highlighting}
\end{Shaded}

\begin{verbatim}
## [1] "H#ll$"    "W$rld"    "H#ll$"    "Univ#rs#"
\end{verbatim}

{\textbf{\texttt{substr()}
}}

\begin{Shaded}
\begin{Highlighting}[]
\NormalTok{x }\OtherTok{\textless{}{-}} \StringTok{"1t345s?"}
\FunctionTok{substr}\NormalTok{(x, }\DecValTok{2}\NormalTok{, }\DecValTok{6}\NormalTok{) }\CommentTok{\# provides strings from 2 to 6}
\end{Highlighting}
\end{Shaded}

\begin{verbatim}
## [1] "t345s"
\end{verbatim}

{\textbf{\texttt{strsplit()}
}}

\begin{Shaded}
\begin{Highlighting}[]
\NormalTok{x }\OtherTok{\textless{}{-}} \StringTok{"R\#Rocks\#!"}
\FunctionTok{strsplit}\NormalTok{(x, }\AttributeTok{split =} \StringTok{"\#"}\NormalTok{)}
\end{Highlighting}
\end{Shaded}

\begin{verbatim}
## [[1]]
## [1] "R"     "Rocks" "!"
\end{verbatim}

\hypertarget{stringr}{%
\section{\texorpdfstring{\texttt{stringr}}{stringr}}\label{stringr}}

\begin{Shaded}
\begin{Highlighting}[]
\FunctionTok{library}\NormalTok{(stringr)}
\end{Highlighting}
\end{Shaded}

\begin{longtable}[]{@{}ccc@{}}
\toprule
Job & stringr & Base R \\
\midrule
\endhead
String concatenation & \texttt{str\_c()} & \texttt{paste()} \\
Number of characters & \texttt{str\_length()} & \texttt{nchar()} \\
Extracts substrings & \texttt{str\_sub()} & \texttt{substr()} \\
Duplicates characters & \texttt{str\_dup()} & \\
Removes leading and trailing whitespace & \texttt{str\_trim()} & \\
Pads a string & \texttt{str\_pad()} & \\
Wraps a string paragraph & \texttt{str\_wrap()} & \\
\bottomrule
\end{longtable}

\hypertarget{regular-expressions}{%
\section{Regular Expressions}\label{regular-expressions}}

A \textbf{regular expression} (or \textbf{regex}) is a set of symbols that describes a text pattern. More formally, a regular expression is a pattern that describes a set of strings.

Regular expressions are a formal language in the sense that the symbols have a defined set of rules to specify the desired patterns.

\hypertarget{stringr-functions-for-regular-expressions}{%
\subsection{\texorpdfstring{\texttt{stringr} Functions for Regular Expressions}{stringr Functions for Regular Expressions}}\label{stringr-functions-for-regular-expressions}}

\begin{longtable}[]{@{}
  >{\centering\arraybackslash}p{(\columnwidth - 2\tabcolsep) * \real{0.25}}
  >{\centering\arraybackslash}p{(\columnwidth - 2\tabcolsep) * \real{0.75}}@{}}
\toprule
Function & Job \\
\midrule
\endhead
\texttt{str\_detect}(str, pattern) & Detects the presence of a pattern and returns TRUE if it is found \\
\texttt{str\_locate}(str, pattern) & Locate the 1st position of a pattern and return a matrix with start \& end.s \\
\texttt{str\_extract}(str, pattern) & Extracts text corresponding to the first match. \\
\texttt{str\_match}(str, pattern) & Extracts capture groups formed by () from the first match. \\
\texttt{str\_split}(str, pattern) & Splits string into pieces and returns a list of character vectors. \\
\bottomrule
\end{longtable}

\hypertarget{sql}{%
\chapter{SQL}\label{sql}}

\hypertarget{create}{%
\section{CREATE}\label{create}}

The general syntax to create a table:

\begin{Shaded}
\begin{Highlighting}[]
\KeywordTok{create} \KeywordTok{table}\NormalTok{ TABLENAME (}
\NormalTok{  COLUMN1 datatype, }
\NormalTok{  COLUMN2 datatype, }
\NormalTok{  COLUMN3 datatype, }
  \OperatorTok{..}\NormalTok{. );}
\end{Highlighting}
\end{Shaded}

To create a table called \texttt{TEST} with two columns - \texttt{ID} of type integer, and \texttt{NAME} of type varchar, we could create it using the following SQL statement:

\begin{Shaded}
\begin{Highlighting}[]
\KeywordTok{create} \KeywordTok{table}\NormalTok{ TEST(}
  \KeywordTok{ID} \DataTypeTok{int}
\NormalTok{  NAME }\DataTypeTok{varchar}\NormalTok{(}\DecValTok{30}\NormalTok{)}
\NormalTok{);}
\end{Highlighting}
\end{Shaded}

To create a table called \texttt{COUNTRY} with an \texttt{ID} column, a two letter country code column \texttt{CCODE}, and a variable length country name column \texttt{NAME}:

\begin{Shaded}
\begin{Highlighting}[]
\KeywordTok{create} \KeywordTok{table}\NormalTok{ COUNTRY(}
    \KeywordTok{ID} \DataTypeTok{int}\NormalTok{,}
\NormalTok{    CCODE }\DataTypeTok{char}\NormalTok{(}\DecValTok{2}\NormalTok{),}
\NormalTok{    NAME }\DataTypeTok{varchar}\NormalTok{(}\DecValTok{60}\NormalTok{)}
\NormalTok{);}
\end{Highlighting}
\end{Shaded}

Sometimes you may see additional keywords in a create table statement:

\begin{Shaded}
\begin{Highlighting}[]
\KeywordTok{create} \KeywordTok{table}\NormalTok{ COUNTRY(}
    \KeywordTok{ID} \DataTypeTok{int} \KeywordTok{NOT} \KeywordTok{NULL}\NormalTok{,}
\NormalTok{    CCODE }\DataTypeTok{char}\NormalTok{(}\DecValTok{2}\NormalTok{),}
\NormalTok{    NAME }\DataTypeTok{varchar}\NormalTok{(}\DecValTok{60}\NormalTok{),}
    \KeywordTok{PRIMARY} \KeywordTok{KEY}\NormalTok{(}\KeywordTok{ID}\NormalTok{)}
\NormalTok{);}
\end{Highlighting}
\end{Shaded}

\begin{itemize}
\item
  In the above example the \texttt{ID} column has the {\textbf{\texttt{NOT\ NULL}}} constraint added after the datatype - meaning that \emph{it cannot contain a NULL or an empty value}.
\item
  If you look at the last row in the create table statement above you will note that we are using \texttt{ID} as a {\textbf{Primary Key}} and the database \textbf{does not allow} Primary Keys to have \textbf{\texttt{NULL}} values. \emph{A Primary Key is a unique identifier in a table, and using Primary Keys can help speed up your queries significantly}.
\item
  If the table you are trying to create already exists in the database, you will get an error indicating table \texttt{XXX.YYY} already exists. To circumvent this error, either create a table with a different name or first \texttt{DROP} the existing table. It is quite common to issue a \texttt{DROP} before doing a \texttt{CREATE} in test and development scenarios.
\end{itemize}

\hypertarget{drop}{%
\section{DROP}\label{drop}}

The general syntax to drop a table:

\begin{Shaded}
\begin{Highlighting}[]
\KeywordTok{drop} \KeywordTok{table}\NormalTok{ TABLENAME;}
\end{Highlighting}
\end{Shaded}

For example, to drop the table COUNTRY, we can use the following code:

\begin{Shaded}
\begin{Highlighting}[]
\KeywordTok{drop} \KeywordTok{table}\NormalTok{ COUNTRY;}
\end{Highlighting}
\end{Shaded}

\hypertarget{alter}{%
\section{ALTER}\label{alter}}

\begin{Shaded}
\begin{Highlighting}[]
\KeywordTok{ALTER} \KeywordTok{TABLE}\NormalTok{ table\_name}
\KeywordTok{ADD} \KeywordTok{COLUMN}\NormalTok{ column\_name data\_type column\_constraint;}

\KeywordTok{ALTER} \KeywordTok{TABLE}\NormalTok{ table\_name}
\KeywordTok{DROP} \KeywordTok{COLUMN}\NormalTok{ column\_name;}

\KeywordTok{ALTER} \KeywordTok{TABLE}\NormalTok{ table\_name}
\KeywordTok{ALTER} \KeywordTok{COLUMN}\NormalTok{ column\_name }\KeywordTok{SET} \KeywordTok{DATA} \KeywordTok{TYPE}\NormalTok{ data\_type;}

\KeywordTok{ALTER} \KeywordTok{TABLE}\NormalTok{ table\_name}
\KeywordTok{RENAME} \KeywordTok{COLUMN}\NormalTok{ current\_column\_name }\KeywordTok{TO}\NormalTok{ new\_column\_name;}
\end{Highlighting}
\end{Shaded}

\hypertarget{truncate}{%
\section{TRUNCATE}\label{truncate}}

\begin{Shaded}
\begin{Highlighting}[]
\KeywordTok{TRUNCATE} \KeywordTok{TABLE}\NormalTok{ table\_name;}
\end{Highlighting}
\end{Shaded}

\hypertarget{guided-exercise-create-table-and-insert-data}{%
\section{Guided Exercise: Create table and insert data}\label{guided-exercise-create-table-and-insert-data}}

You will to create two tables

\begin{enumerate}
\def\labelenumi{\arabic{enumi}.}
\item
  \texttt{PETSALE}
\item
  \texttt{PET}.
\end{enumerate}

\begin{Shaded}
\begin{Highlighting}[]
\KeywordTok{CREATE} \KeywordTok{TABLE}\NormalTok{ PETSALE (}
    \KeywordTok{ID} \DataTypeTok{INTEGER} \KeywordTok{NOT} \KeywordTok{NULL}\NormalTok{,}
\NormalTok{    PET }\DataTypeTok{CHAR}\NormalTok{(}\DecValTok{20}\NormalTok{),}
\NormalTok{    SALEPRICE }\DataTypeTok{DECIMAL}\NormalTok{(}\DecValTok{6}\NormalTok{,}\DecValTok{2}\NormalTok{),}
\NormalTok{    PROFIT }\DataTypeTok{DECIMAL}\NormalTok{(}\DecValTok{6}\NormalTok{,}\DecValTok{2}\NormalTok{),}
\NormalTok{    SALEDATE }\DataTypeTok{DATE}
\NormalTok{    );}
    
\KeywordTok{CREATE} \KeywordTok{TABLE}\NormalTok{ PET (}
    \KeywordTok{ID} \DataTypeTok{INTEGER} \KeywordTok{NOT} \KeywordTok{NULL}\NormalTok{,}
\NormalTok{    ANIMAL }\DataTypeTok{VARCHAR}\NormalTok{(}\DecValTok{20}\NormalTok{),}
\NormalTok{    QUANTITY }\DataTypeTok{INTEGER}
\NormalTok{    );}
\end{Highlighting}
\end{Shaded}

{\emph{Now insert some records into the two newly created tables and show all the records of the two tables. }}

\begin{Shaded}
\begin{Highlighting}[]
\KeywordTok{INSERT} \KeywordTok{INTO}\NormalTok{ PETSALE }\KeywordTok{VALUES}
\NormalTok{    (}\DecValTok{1}\NormalTok{,}\StringTok{\textquotesingle{}Cat\textquotesingle{}}\NormalTok{,}\FloatTok{450.09}\NormalTok{,}\FloatTok{100.47}\NormalTok{,}\StringTok{\textquotesingle{}2018{-}05{-}29\textquotesingle{}}\NormalTok{),}
\NormalTok{    (}\DecValTok{2}\NormalTok{,}\StringTok{\textquotesingle{}Dog\textquotesingle{}}\NormalTok{,}\FloatTok{666.66}\NormalTok{,}\FloatTok{150.76}\NormalTok{,}\StringTok{\textquotesingle{}2018{-}06{-}01\textquotesingle{}}\NormalTok{),}
\NormalTok{    (}\DecValTok{3}\NormalTok{,}\StringTok{\textquotesingle{}Parrot\textquotesingle{}}\NormalTok{,}\FloatTok{50.00}\NormalTok{,}\FloatTok{8.9}\NormalTok{,}\StringTok{\textquotesingle{}2018{-}06{-}04\textquotesingle{}}\NormalTok{),}
\NormalTok{    (}\DecValTok{4}\NormalTok{,}\StringTok{\textquotesingle{}Hamster\textquotesingle{}}\NormalTok{,}\FloatTok{60.60}\NormalTok{,}\DecValTok{12}\NormalTok{,}\StringTok{\textquotesingle{}2018{-}06{-}11\textquotesingle{}}\NormalTok{),}
\NormalTok{    (}\DecValTok{5}\NormalTok{,}\StringTok{\textquotesingle{}Goldfish\textquotesingle{}}\NormalTok{,}\FloatTok{48.48}\NormalTok{,}\FloatTok{3.5}\NormalTok{,}\StringTok{\textquotesingle{}2018{-}06{-}14\textquotesingle{}}\NormalTok{);}
    
\KeywordTok{INSERT} \KeywordTok{INTO}\NormalTok{ PET }\KeywordTok{VALUES}
\NormalTok{    (}\DecValTok{1}\NormalTok{,}\StringTok{\textquotesingle{}Cat\textquotesingle{}}\NormalTok{,}\DecValTok{3}\NormalTok{),}
\NormalTok{    (}\DecValTok{2}\NormalTok{,}\StringTok{\textquotesingle{}Dog\textquotesingle{}}\NormalTok{,}\DecValTok{4}\NormalTok{),}
\NormalTok{    (}\DecValTok{3}\NormalTok{,}\StringTok{\textquotesingle{}Hamster\textquotesingle{}}\NormalTok{,}\DecValTok{2}\NormalTok{);}
    
\KeywordTok{SELECT} \OperatorTok{*} \KeywordTok{FROM}\NormalTok{ PETSALE;}
\KeywordTok{SELECT} \OperatorTok{*} \KeywordTok{FROM}\NormalTok{ PET;}
\end{Highlighting}
\end{Shaded}

\hypertarget{guided-exercise-use-the-alter-statement-to-add-delete-or-modify-columns-in-two-of-the-existing-tables-created-in-the-previous-exercise.}{%
\section{\texorpdfstring{Guided Exercise: Use the \texttt{ALTER} statement to add, delete, or modify columns in two of the existing tables created in the previous exercise.}{Guided Exercise: Use the ALTER statement to add, delete, or modify columns in two of the existing tables created in the previous exercise.}}\label{guided-exercise-use-the-alter-statement-to-add-delete-or-modify-columns-in-two-of-the-existing-tables-created-in-the-previous-exercise.}}

{\emph{Add a new \texttt{QUANTITY} column to the \texttt{PETSALE} table and show the altered table.}}

\begin{Shaded}
\begin{Highlighting}[]
\KeywordTok{ALTER} \KeywordTok{TABLE}\NormalTok{ PETSALE}
\KeywordTok{ADD} \KeywordTok{COLUMN}\NormalTok{ QUANTITY }\DataTypeTok{INTEGER}\NormalTok{;}

\KeywordTok{SELECT} \OperatorTok{*} \KeywordTok{FROM}\NormalTok{ PETSALE;}
\end{Highlighting}
\end{Shaded}

{\emph{Now update the newly added \texttt{QUANTITY} column of the \texttt{PETSALE} table with some values and show all the records of the table.
}}

\begin{Shaded}
\begin{Highlighting}[]
\KeywordTok{UPDATE}\NormalTok{ PETSALE }\KeywordTok{SET}\NormalTok{ QUANTITY }\OperatorTok{=} \DecValTok{9} \KeywordTok{WHERE} \KeywordTok{ID} \OperatorTok{=} \DecValTok{1}\NormalTok{;}
\KeywordTok{UPDATE}\NormalTok{ PETSALE }\KeywordTok{SET}\NormalTok{ QUANTITY }\OperatorTok{=} \DecValTok{3} \KeywordTok{WHERE} \KeywordTok{ID} \OperatorTok{=} \DecValTok{2}\NormalTok{;}
\KeywordTok{UPDATE}\NormalTok{ PETSALE }\KeywordTok{SET}\NormalTok{ QUANTITY }\OperatorTok{=} \DecValTok{2} \KeywordTok{WHERE} \KeywordTok{ID} \OperatorTok{=} \DecValTok{3}\NormalTok{;}
\KeywordTok{UPDATE}\NormalTok{ PETSALE }\KeywordTok{SET}\NormalTok{ QUANTITY }\OperatorTok{=} \DecValTok{6} \KeywordTok{WHERE} \KeywordTok{ID} \OperatorTok{=} \DecValTok{4}\NormalTok{;}
\KeywordTok{UPDATE}\NormalTok{ PETSALE }\KeywordTok{SET}\NormalTok{ QUANTITY }\OperatorTok{=} \DecValTok{24} \KeywordTok{WHERE} \KeywordTok{ID} \OperatorTok{=} \DecValTok{5}\NormalTok{;}

\KeywordTok{SELECT} \OperatorTok{*} \KeywordTok{FROM}\NormalTok{ PETSALE;}
\end{Highlighting}
\end{Shaded}

{\emph{Delete the \texttt{PROFIT} column from the \texttt{PETSALE} table and show the altered table.
}}

\begin{Shaded}
\begin{Highlighting}[]
\KeywordTok{ALTER} \KeywordTok{TABLE}\NormalTok{ PETSALE}
\KeywordTok{DROP} \KeywordTok{COLUMN}\NormalTok{ PROFIT;}

\KeywordTok{SELECT} \OperatorTok{*} \KeywordTok{FROM}\NormalTok{ PETSALE;}
\end{Highlighting}
\end{Shaded}

{\emph{Change the data type to \texttt{VARCHAR(20)} type of the column \texttt{PET} of the table \texttt{PETSALE} and show the altered table.
}}

\begin{Shaded}
\begin{Highlighting}[]
\KeywordTok{ALTER} \KeywordTok{TABLE}\NormalTok{ PETSALE}
\KeywordTok{ALTER} \KeywordTok{COLUMN}\NormalTok{ PET }\KeywordTok{SET} \KeywordTok{DATA} \KeywordTok{TYPE} \DataTypeTok{VARCHAR}\NormalTok{(}\DecValTok{20}\NormalTok{);}

\KeywordTok{SELECT} \OperatorTok{*} \KeywordTok{FROM}\NormalTok{ PETSALE;}
\end{Highlighting}
\end{Shaded}

If you are using IBM db2:
Now verify if the data type of the column PET of the table PETSALE changed to \texttt{VARCHAR(20)} type or not. Click on the 3 bar menu icon in the top left corner and click Explore \textgreater{} Tables. Find the \texttt{PETSALE} table from Schemas by clicking Select All. Click on the \texttt{PETSALE} table to open the Table Definition page of the table. Here, you can see all the current data type of the columns of the \texttt{PETSALE} table.

{\emph{Rename the column PET to ANIMAL of the PETSALE table and show the altered table.
}}

\begin{Shaded}
\begin{Highlighting}[]
\KeywordTok{ALTER} \KeywordTok{TABLE}\NormalTok{ PETSALE}
\KeywordTok{RENAME} \KeywordTok{COLUMN}\NormalTok{ PET }\KeywordTok{TO}\NormalTok{ ANIMAL;}

\KeywordTok{SELECT} \OperatorTok{*} \KeywordTok{FROM}\NormalTok{ PETSALE;}
\end{Highlighting}
\end{Shaded}

\hypertarget{guided-exercise-truncate}{%
\section{Guided Exercise: TRUNCATE}\label{guided-exercise-truncate}}

In this exercise, you will use the \texttt{TRUNCATE} statement to remove all rows from an existing table created in exercise 1 without deleting the table itself.

{\emph{Remove all rows from the PET table and show the empty table.
}}

\begin{Shaded}
\begin{Highlighting}[]
\KeywordTok{TRUNCATE} \KeywordTok{TABLE}\NormalTok{ PET }\KeywordTok{IMMEDIATE}\NormalTok{;}
\KeywordTok{SELECT} \OperatorTok{*} \KeywordTok{FROM}\NormalTok{ PET;}
\end{Highlighting}
\end{Shaded}

\hypertarget{guided-exercise-drop}{%
\section{Guided Exercise: DROP}\label{guided-exercise-drop}}

In this exercise, you will use the \texttt{DROP} statement to delete an existing table created in the previous exercise.

{\emph{Delete the PET table and verify if the table still exists or not (SELECT statement won't work if a table doesn't exist).
}}

\begin{Shaded}
\begin{Highlighting}[]
\KeywordTok{DROP} \KeywordTok{TABLE}\NormalTok{ PET;}
\KeywordTok{SELECT} \OperatorTok{*} \KeywordTok{FROM}\NormalTok{ PET;}
\end{Highlighting}
\end{Shaded}

\hypertarget{exercise-string-patterns}{%
\section{Exercise: String Patterns}\label{exercise-string-patterns}}

In this exercise, you will go through some SQL problems on String Patterns.

Here is \texttt{EMPLOYEES} table.

\begin{tabular}{l|l|l|r|l}
\hline
EMP\_ID & F\_NAME & L\_NAME & SSN & B\_DATE\\
\hline
E1001 & John & Thomas & 123456 & 1976-01-09\\
\hline
E1002 & Alice & James & 123457 & 1972-07-31\\
\hline
E1003 & Steve & Wells & 123458 & 1980-08-10\\
\hline
E1004 & Santosh & Kumar & 123459 & 1985-07-20\\
\hline
E1005 & Ahmed & Hussain & 123410 & 1981-01-04\\
\hline
E1006 & Nancy & Allen & 123411 & 1978-02-06\\
\hline
E1007 & Mary & Thomas & 123412 & 1975-05-05\\
\hline
E1008 & Bharath & Gupta & 123413 & 1985-05-06\\
\hline
E1009 & Andrea & Jones & 123414 & 1990-07-09\\
\hline
E1010 & Ann & Jacob & 123415 & 1982-03-30\\
\hline
\end{tabular}

\begin{tabular}{l|l|r|r|r|r}
\hline
SEX & ADDRESS & JOB\_ID & SALARY & MANAGER\_ID & DEP\_ID\\
\hline
M & 5631 Rice, OakPark,IL & 100 & 100000 & 30001 & 2\\
\hline
F & 980 Berry ln, Elgin,IL & 200 & 80000 & 30002 & 5\\
\hline
M & 291 Springs, Gary,IL & 300 & 50000 & 30002 & 5\\
\hline
M & 511 Aurora Av, Aurora,IL & 400 & 60000 & 30004 & 5\\
\hline
M & 216 Oak Tree, Geneva,IL & 500 & 70000 & 30001 & 2\\
\hline
F & 111 Green Pl, Elgin,IL & 600 & 90000 & 30001 & 2\\
\hline
F & 100 Rose Pl, Gary,IL & 650 & 65000 & 30003 & 7\\
\hline
M & 145 Berry Ln, Naperville,IL & 660 & 65000 & 30003 & 7\\
\hline
F & 120 Fall Creek, Gary,IL & 234 & 70000 & 30003 & 7\\
\hline
F & 111 Britany Springs,Elgin,IL & 220 & 70000 & 30004 & 5\\
\hline
\end{tabular}

\hypertarget{retrieve-all-employees-whose-address-is-in-elginil.}{%
\subsection{\texorpdfstring{\emph{Retrieve all employees whose address is in Elgin,IL}.}{Retrieve all employees whose address is in Elgin,IL.}}\label{retrieve-all-employees-whose-address-is-in-elginil.}}

Click here for the solution

\begin{Shaded}
\begin{Highlighting}[]
\KeywordTok{SELECT}\NormalTok{ F\_NAME , L\_NAME}
\KeywordTok{FROM}\NormalTok{ EMPLOYEES}
\KeywordTok{WHERE}\NormalTok{ ADDRESS }\KeywordTok{LIKE} \StringTok{\textquotesingle{}\%Elgin,IL\%\textquotesingle{}}\NormalTok{;}
\end{Highlighting}
\end{Shaded}

\hypertarget{retrieve-all-employees-who-were-born-during-the-1970s..}{%
\subsection{\texorpdfstring{\emph{Retrieve all employees who were born during the 1970's.}.}{Retrieve all employees who were born during the 1970's..}}\label{retrieve-all-employees-who-were-born-during-the-1970s..}}

Click here for the solution

\begin{Shaded}
\begin{Highlighting}[]
\KeywordTok{SELECT}\NormalTok{ F\_NAME , L\_NAME}
\KeywordTok{FROM}\NormalTok{ EMPLOYEES}
\KeywordTok{WHERE}\NormalTok{ B\_DATE }\KeywordTok{LIKE} \StringTok{\textquotesingle{}197\%\textquotesingle{}}\NormalTok{;}
\end{Highlighting}
\end{Shaded}

\hypertarget{retrieve-all-employees-in-department-5-whose-salary-is-between-60000-and-70000..}{%
\subsection{\texorpdfstring{\emph{Retrieve all employees in department 5 whose salary is between 60000 and 70000.}.}{Retrieve all employees in department 5 whose salary is between 60000 and 70000..}}\label{retrieve-all-employees-in-department-5-whose-salary-is-between-60000-and-70000..}}

Click here for the solution

\begin{Shaded}
\begin{Highlighting}[]
\KeywordTok{SELECT}\NormalTok{ F\_NAME , L\_NAME}
\KeywordTok{FROM}\NormalTok{ EMPLOYEES}
\KeywordTok{WHERE}\NormalTok{ DEP\_ID }\OperatorTok{=} \DecValTok{5} \KeywordTok{and}\NormalTok{ (SALARY }\KeywordTok{BETWEEN} \DecValTok{60000} \KeywordTok{AND} \DecValTok{70000}\NormalTok{);}
\CommentTok{{-}{-}Notice the "=" and "and"}
\end{Highlighting}
\end{Shaded}

\hypertarget{exercise-sorting}{%
\section{Exercise: Sorting}\label{exercise-sorting}}

\hypertarget{retrieve-a-list-of-employees-ordered-by-department-id..}{%
\subsection{\texorpdfstring{\emph{Retrieve a list of employees ordered by department ID.}.}{Retrieve a list of employees ordered by department ID..}}\label{retrieve-a-list-of-employees-ordered-by-department-id..}}

Click here for the solution

\begin{Shaded}
\begin{Highlighting}[]
\KeywordTok{SELECT}\NormalTok{ F\_NAME, L\_NAME, DEP\_ID }
\KeywordTok{FROM}\NormalTok{ EMPLOYEES}
\KeywordTok{ORDER} \KeywordTok{BY}\NormalTok{ DEP\_ID;}
\end{Highlighting}
\end{Shaded}

\hypertarget{retrieve-a-list-of-employees-ordered-in-descending-order-by-department-id-and-within-each-department-ordered-alphabetically-in-descending-order-by-last-name..}{%
\subsection{\texorpdfstring{\emph{Retrieve a list of employees ordered in descending order by department ID and within each department ordered alphabetically in descending order by last name.}.}{Retrieve a list of employees ordered in descending order by department ID and within each department ordered alphabetically in descending order by last name..}}\label{retrieve-a-list-of-employees-ordered-in-descending-order-by-department-id-and-within-each-department-ordered-alphabetically-in-descending-order-by-last-name..}}

Click here for the solution

\begin{Shaded}
\begin{Highlighting}[]
\KeywordTok{SELECT}\NormalTok{ F\_NAME, L\_NAME, DEP\_ID }
\KeywordTok{FROM}\NormalTok{ EMPLOYEES}
\KeywordTok{ORDER} \KeywordTok{BY}\NormalTok{ DEP\_ID }\KeywordTok{DESC}\NormalTok{, L\_NAME }\KeywordTok{DESC}\NormalTok{;}
\end{Highlighting}
\end{Shaded}

\hypertarget{in-the-previous-problem-use-department-name-instead-of-department-id.-retrieve-a-list-of-employees-ordered-by-department-name-and-within-each-department-ordered-alphabetically-in-descending-order-by-last-name..}{%
\subsection{\texorpdfstring{\emph{In the previous problem, use department name instead of department ID. Retrieve a list of employees ordered by department name, and within each department ordered alphabetically in descending order by last name.}.}{In the previous problem, use department name instead of department ID. Retrieve a list of employees ordered by department name, and within each department ordered alphabetically in descending order by last name..}}\label{in-the-previous-problem-use-department-name-instead-of-department-id.-retrieve-a-list-of-employees-ordered-by-department-name-and-within-each-department-ordered-alphabetically-in-descending-order-by-last-name..}}

Here is the \texttt{DEPARTMENTS} table.

\begin{tabular}{r|l|r|l}
\hline
DEPT\_ID\_DEP & DEP\_NAME & MANAGER\_ID & LOC\_ID\\
\hline
2 & Architect Group & 30001 & L0001\\
\hline
5 & Software Group & 30002 & L0002\\
\hline
7 & Design Team & 30003 & L0003\\
\hline
\end{tabular}

Click here for the solution

\begin{Shaded}
\begin{Highlighting}[]
\KeywordTok{SELECT}\NormalTok{ D.DEP\_NAME , E.F\_NAME, E.L\_NAME}
\KeywordTok{FROM}\NormalTok{ EMPLOYEES }\KeywordTok{as}\NormalTok{ E, DEPARTMENTS }\KeywordTok{as}\NormalTok{ D}
\KeywordTok{WHERE}\NormalTok{ E.DEP\_ID }\OperatorTok{=}\NormalTok{ D.DEPT\_ID\_DEP}
\KeywordTok{ORDER} \KeywordTok{BY}\NormalTok{ D.DEP\_NAME, E.L\_NAME }\KeywordTok{DESC}\NormalTok{;}
\end{Highlighting}
\end{Shaded}

\begin{quote}
In the SQL Query above, \texttt{D} and \texttt{E} are aliases for the table names. Once you define an alias like \texttt{D} in your query, you can simply write \texttt{D.COLUMN\_NAME} rather than the full form \texttt{DEPARTMENTS.COLUMN\_NAME}.
\end{quote}

\hypertarget{exercise-3-grouping}{%
\section{Exercise 3: Grouping}\label{exercise-3-grouping}}

\hypertarget{for-each-department-id-retrieve-the-number-of-employees-in-the-department..}{%
\subsection{\texorpdfstring{\emph{For each department ID retrieve the number of employees in the department.}.}{For each department ID retrieve the number of employees in the department..}}\label{for-each-department-id-retrieve-the-number-of-employees-in-the-department..}}

Click here for the solution

\begin{Shaded}
\begin{Highlighting}[]
\KeywordTok{SELECT}\NormalTok{ DEP\_ID, }\FunctionTok{COUNT}\NormalTok{(}\OperatorTok{*}\NormalTok{)}
\KeywordTok{FROM}\NormalTok{ EMPLOYEES}
\KeywordTok{GROUP} \KeywordTok{BY}\NormalTok{ DEP\_ID;}
\end{Highlighting}
\end{Shaded}

\hypertarget{for-each-department-retrieve-the-number-of-employees-in-the-department-and-the-average-employee-salary-in-the-department..}{%
\subsection{\texorpdfstring{\emph{For each department retrieve the number of employees in the department, and the average employee salary in the department.}.}{For each department retrieve the number of employees in the department, and the average employee salary in the department..}}\label{for-each-department-retrieve-the-number-of-employees-in-the-department-and-the-average-employee-salary-in-the-department..}}

Click here for the solution

\begin{Shaded}
\begin{Highlighting}[]
\KeywordTok{SELECT}\NormalTok{ DEP\_ID, }\FunctionTok{COUNT}\NormalTok{(}\OperatorTok{*}\NormalTok{), }\FunctionTok{AVG}\NormalTok{(SALARY)}
\KeywordTok{FROM}\NormalTok{ EMPLOYEES}
\KeywordTok{GROUP} \KeywordTok{BY}\NormalTok{ DEP\_ID;}
\end{Highlighting}
\end{Shaded}

\hypertarget{label-the-computed-columns-in-the-result-set-of-the-last-sql-problem-as-num_employees-and-avg_salary..}{%
\subsection{\texorpdfstring{\emph{Label the computed columns in the result set of the last SQL problem as NUM\_EMPLOYEES and AVG\_SALARY.}.}{Label the computed columns in the result set of the last SQL problem as NUM\_EMPLOYEES and AVG\_SALARY..}}\label{label-the-computed-columns-in-the-result-set-of-the-last-sql-problem-as-num_employees-and-avg_salary..}}

Click here for the solution

\begin{Shaded}
\begin{Highlighting}[]
\KeywordTok{SELECT}\NormalTok{ DEP\_ID, }\FunctionTok{COUNT}\NormalTok{(}\OperatorTok{*}\NormalTok{) }\KeywordTok{AS} \OtherTok{"NUM\_EMPLOYEES"}\NormalTok{, }\FunctionTok{AVG}\NormalTok{(SALARY) }\KeywordTok{AS} \OtherTok{"AVG\_SALARY"}
\KeywordTok{FROM}\NormalTok{ EMPLOYEES}
\KeywordTok{GROUP} \KeywordTok{BY}\NormalTok{ DEP\_ID;}
\end{Highlighting}
\end{Shaded}

\hypertarget{in-the-previous-sql-problem-order-the-result-set-by-average-salary..}{%
\subsection{\texorpdfstring{\emph{In the previous SQL problem , order the result set by Average Salary.}.}{In the previous SQL problem , order the result set by Average Salary..}}\label{in-the-previous-sql-problem-order-the-result-set-by-average-salary..}}

Click here for the solution

\begin{Shaded}
\begin{Highlighting}[]
\KeywordTok{SELECT}\NormalTok{ DEP\_ID, }\FunctionTok{COUNT}\NormalTok{(}\OperatorTok{*}\NormalTok{) }\KeywordTok{AS} \OtherTok{"NUM\_EMPLOYEES"}\NormalTok{, }\FunctionTok{AVG}\NormalTok{(SALARY) }\KeywordTok{AS} \OtherTok{"AVG\_SALARY"}
\KeywordTok{FROM}\NormalTok{ EMPLOYEES}
\KeywordTok{GROUP} \KeywordTok{BY}\NormalTok{ DEP\_ID}
\KeywordTok{ORDER} \KeywordTok{BY}\NormalTok{ AVG\_SALARY;}
\end{Highlighting}
\end{Shaded}

\hypertarget{in-sql-problem-4-exercise-3-problem-4-limit-the-result-to-departments-with-fewer-than-4-employees..}{%
\subsection{\texorpdfstring{\emph{In SQL problem 4 (Exercise 3 Problem 4), limit the result to departments with fewer than 4 employees.}.}{In SQL problem 4 (Exercise 3 Problem 4), limit the result to departments with fewer than 4 employees..}}\label{in-sql-problem-4-exercise-3-problem-4-limit-the-result-to-departments-with-fewer-than-4-employees..}}

Click here for the solution

\begin{Shaded}
\begin{Highlighting}[]
\KeywordTok{SELECT}\NormalTok{ DEP\_ID, }\FunctionTok{COUNT}\NormalTok{(}\OperatorTok{*}\NormalTok{) }\KeywordTok{AS} \OtherTok{"NUM\_EMPLOYEES"}\NormalTok{, }\FunctionTok{AVG}\NormalTok{(SALARY) }\KeywordTok{AS} \OtherTok{"AVG\_SALARY"}
\KeywordTok{FROM}\NormalTok{ EMPLOYEES}
\KeywordTok{GROUP} \KeywordTok{BY}\NormalTok{ DEP\_ID}
\KeywordTok{HAVING} \FunctionTok{count}\NormalTok{(}\OperatorTok{*}\NormalTok{) }\OperatorTok{\textless{}} \DecValTok{4}
\KeywordTok{ORDER} \KeywordTok{BY}\NormalTok{ AVG\_SALARY;}
\end{Highlighting}
\end{Shaded}


  \bibliography{book.bib,packages.bib}

\end{document}
